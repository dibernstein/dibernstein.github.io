 \documentclass[letterpaper,11pt]{amsart}
\textwidth=16.00cm 
\textheight=22.00cm 
\topmargin=0.00cm
\oddsidemargin=0.00cm 
\evensidemargin=0.00cm 
\headheight=0cm 
\headsep=0.5cm

\textheight=610pt

\usepackage{latexsym,amsthm,amssymb,epsfig,url}%eufrak
%\usepackage[sumlimits]{amsmath}

% \usepackage[centredisplay,PostScript=dvips]{diagrams}
%\usepackage[dvips]{color}
%\usepackage[mathscr]{eucal}mathrsfs

% Theorem Formatting Commands.
\theoremstyle{plain}
\newtheorem{thm}{Theorem}
%\newtheorem{lemma}{Lemma}[section]
\newtheorem{lemma}[thm]{Lemma}
\newtheorem{prop}[thm]{Proposition}
\newtheorem{cor}[thm]{Corollary}
\newtheorem{conj}[thm]{Conjecture}
\newtheorem*{thm*}{Theorem}
\newtheorem*{lemma*}{Lemma}
\newtheorem*{prop*}{Proposition}
\newtheorem*{cor*}{Corollary}
\newtheorem*{conj*}{Conjecture}

\theoremstyle{definition}
\newtheorem{defn}[thm]{Definition}
\newtheorem*{defn*}{Definition}
\newtheorem{ex}[thm]{Example}
\newtheorem{pr}{Problem}
\newtheorem{alg}[thm]{Algorithm}
\newtheorem{ques}[thm]{Question}

\theoremstyle{remark}
\newtheorem*{rmk}{Remark}
\newtheorem*{obs}{Observation}

\usepackage{color}
\newcommand{\dbfeedback}[1]{{\color{red} DB Feedback: #1}}



\DeclareMathOperator{\gl}{GL}
\DeclareMathOperator{\aut}{Aut}



\begin{document}

\Large

\begin{center}
{\bf Math 7110 -- Homework  7 --  Due:  November 8, 2021}
\end{center}

\normalsize


\medskip

\noindent {\bf Practice Problems:}
\begin{pr}
    Dummit and Foote, section 7.4: 8, 9
\end{pr}

\noindent {\bf Test practice:}

\begin{pr}
    Determine whether or not the following rings are fields:
    \begin{enumerate}
        \item $\mathbb{Z}[x]/(x+1)$
        \item $\mathbb{R}[x]/(x^2-1)$
        \item $\mathbb{R}[x]/(x^2+1)$
        \item $\mathbb{Z}_2[x]/(x^2+x+1)$ 
    \end{enumerate}
\end{pr}


% \begin{pr}
%     Let $R$ be the ring of all functions $[0,1]\rightarrow \mathbb{R}$ and let $S$ be the ring of all \emph{continuous} functions $[0,1]\rightarrow \mathbb{R}$.
%     \begin{enumerate}
%         \item Is $\{f \in R: f\left( \frac{1}{2} \right) = 0 \}$ an ideal of $R$? Is it a principal ideal?
%         \item Is $\{f \in S: f\left( \frac{1}{2} \right) = 0 \}$ an ideal of $S$? Is it a principal ideal?
%     \end{enumerate}
% \end{pr}

\bigskip

Type solutions to the following problems in \LaTeX, and email the tex and PDF files to me at \url{dbernstein1@tulane.edu} by 10am on the due date.
Please title them as [lastname].tex and [lastname].pdf.
When preparing your solutions, you must follow the rules as laid out in the course syllabus.

\vspace{.5cm}

\noindent {\bf Graded Problems:}

\begin{pr}
    Assume $R$ is commutative and let $I \subseteq R$ be an ideal.
    Define
    \[
        {\rm rad}(I) = \{r \in R : r^n \in I {\rm \ for \ some \ } n \ge 0\}. 
    \]
    An ideal is said to be \emph{radical} if $I = {\rm rad}(I)$.
    \begin{enumerate}
        \item Prove that ${\rm rad}(I)$ is an ideal of $R$
        \item Prove that prime ideals are radical
        \item Under what conditions on $n$ is $(n)\subseteq \mathbb{Z}$ a radical ideal?
    \end{enumerate}
\end{pr}

\begin{pr}
    A \emph{local ring} is a commutative ring with a unique maximal ideal.
    Let $R$ be a local ring with maximal ideal $M$.
    \begin{enumerate}
        \item Prove that every $f \in R \setminus M$ is a unit
        \item Prove that if $R$ is commutative and has a $1$, then if the set $M$ of non-units forms an ideal,
        then $R$ is local with maximal ideal $M$
        \item Let $R$ be the ring of rational numbers with odd denominator. Prove that $R$ is local with maximal ideal $(2)$.
    \end{enumerate}
\end{pr}


\end{document}