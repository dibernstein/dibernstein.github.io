 \documentclass[letterpaper,11pt]{amsart}
\textwidth=16.00cm 
\textheight=22.00cm 
\topmargin=0.00cm
\oddsidemargin=0.00cm 
\evensidemargin=0.00cm 
\headheight=0cm 
\headsep=0.5cm

\textheight=610pt

\usepackage{latexsym,amsthm,amssymb,epsfig,url}%eufrak
%\usepackage[sumlimits]{amsmath}

% \usepackage[centredisplay,PostScript=dvips]{diagrams}
%\usepackage[dvips]{color}
%\usepackage[mathscr]{eucal}mathrsfs

% Theorem Formatting Commands.
\theoremstyle{plain}
\newtheorem{thm}{Theorem}
%\newtheorem{lemma}{Lemma}[section]
\newtheorem{lemma}[thm]{Lemma}
\newtheorem{prop}[thm]{Proposition}
\newtheorem{cor}[thm]{Corollary}
\newtheorem{conj}[thm]{Conjecture}
\newtheorem*{thm*}{Theorem}
\newtheorem*{lemma*}{Lemma}
\newtheorem*{prop*}{Proposition}
\newtheorem*{cor*}{Corollary}
\newtheorem*{conj*}{Conjecture}

\theoremstyle{definition}
\newtheorem{defn}[thm]{Definition}
\newtheorem*{defn*}{Definition}
\newtheorem{ex}[thm]{Example}
\newtheorem{pr}{Problem}
\newtheorem{alg}[thm]{Algorithm}
\newtheorem{ques}[thm]{Question}

\theoremstyle{remark}
\newtheorem*{rmk}{Remark}
\newtheorem*{obs}{Observation}



\DeclareMathOperator{\gl}{GL}
\DeclareMathOperator{\aut}{Aut}



\begin{document}

\Large

\begin{center}
{\bf Math 7110 -- Homework  3 --  Due:  Oct 1, 2021}
\end{center}

\normalsize


\medskip

\noindent {\bf Practice Problems:}

\begin{defn*}
    Let $H$ and $K$ be subgroups of a group and define
    \[
        HK = \{hk: h \in H, k \in K\}.
    \]
\end{defn*}

The following problem walks you through the results that are proved at the end of Section~3.2 in the Dummit and Foote. Feel free to consult the text.

\begin{pr}
    Let $H$ and $K$ be subgroups of a group $G$.
    \begin{enumerate}
        \item Is $HK$ always a subgroup of $G$? Prove or give a counterexample (spoiler alert: answer is below).
        \item Assume $H$ and $K$ are finite and prove the following
        \[
            |HK| = \frac{|H||K|}{|H\cap K|}.
        \]
        \item Prove that $HK$ is a subgroup of $G$ if and only if $HK = KH$.
        \item Does $HK = KH$ mean that every element of $H$ commutes with every element of $K$? (Hint: find subgroups $H$ and $K$ of $D_8$ such that $D_8 = HK$).
    \end{enumerate}
\end{pr}

\begin{pr}
     Read the statement and proofs of the second and third isomorphism theorems in Dummit and Foote.
\end{pr}

\bigskip

Type solutions to the following problems in \LaTeX, and email the tex and PDF files to me at \url{dbernstein1@tulane.edu} by 10am on the due date.
Please title them as [lastname].tex and [lastname].pdf.
When preparing your solutions, you must follow the rules as laid out in the course syllabus.

\vspace{.5cm}

\noindent {\bf Graded Problems:} 


\begin{pr}
    In this problem, you will prove the Jordan-H\"older Theorem.
    Let $G$ be a finite nontrivial group.
    \begin{enumerate}
        \item Prove that $G$ has a composition series.
        \item Assume that $G$ has two composition series
        \[
            1 = N_0 \trianglelefteq \dots \trianglelefteq N_r = G \qquad {\rm and} \qquad 1 = M_0 \trianglelefteq M_1 \trianglelefteq M_2 = G.
        \]
        Show that $r = 2$ and that the list of composition factors is the same (use the second isomorphism theorem).
        \item Prove the following by induction on $\min\{r,s\}$: If
                \[
            1 = N_0 \trianglelefteq \dots \trianglelefteq N_r = G \qquad {\rm and} \qquad 1 = M_0 \trianglelefteq \dots \trianglelefteq M_s = G
        \]
        are composition series for $G$, then $r = s$ and there is some permutation $\pi$ of $\{1,\dots,r\}$ such that
        \[
            M_{\pi(i)} / M_{\pi(i-1)} \cong N_{i} / N_{i-1} \qquad {\rm for} \qquad 1 \le i \le r
        \]
        (hint: apply the induction hypothesis to $H := N_{r-1} \cap M_{s-1}$).
    \end{enumerate}
\end{pr}

\begin{pr}
    Solve the following problems.
    \begin{enumerate}
        \item Find all finite groups with exactly two conjugacy classes.
        \item Let $n$ be odd.
        Show that the set of $n$-cycles consists of two equally-sized conjugacy classes of $A_n$.
    \end{enumerate}
\end{pr}




\end{document}