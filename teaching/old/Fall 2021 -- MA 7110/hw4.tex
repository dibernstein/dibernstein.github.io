 \documentclass[letterpaper,11pt]{amsart}
\textwidth=16.00cm 
\textheight=22.00cm 
\topmargin=0.00cm
\oddsidemargin=0.00cm 
\evensidemargin=0.00cm 
\headheight=0cm 
\headsep=0.5cm

\textheight=610pt

\usepackage{latexsym,amsthm,amssymb,epsfig,url}%eufrak
%\usepackage[sumlimits]{amsmath}

% \usepackage[centredisplay,PostScript=dvips]{diagrams}
%\usepackage[dvips]{color}
%\usepackage[mathscr]{eucal}mathrsfs

% Theorem Formatting Commands.
\theoremstyle{plain}
\newtheorem{thm}{Theorem}
%\newtheorem{lemma}{Lemma}[section]
\newtheorem{lemma}[thm]{Lemma}
\newtheorem{prop}[thm]{Proposition}
\newtheorem{cor}[thm]{Corollary}
\newtheorem{conj}[thm]{Conjecture}
\newtheorem*{thm*}{Theorem}
\newtheorem*{lemma*}{Lemma}
\newtheorem*{prop*}{Proposition}
\newtheorem*{cor*}{Corollary}
\newtheorem*{conj*}{Conjecture}

\theoremstyle{definition}
\newtheorem{defn}[thm]{Definition}
\newtheorem*{defn*}{Definition}
\newtheorem{ex}[thm]{Example}
\newtheorem{pr}{Problem}
\newtheorem{alg}[thm]{Algorithm}
\newtheorem{ques}[thm]{Question}

\theoremstyle{remark}
\newtheorem*{rmk}{Remark}
\newtheorem*{obs}{Observation}

\usepackage{color}
\newcommand{\dbfeedback}[1]{{\color{red} DB Feedback: #1}}



\DeclareMathOperator{\gl}{GL}
\DeclareMathOperator{\aut}{Aut}



\begin{document}

\Large

\begin{center}
{\bf Math 7110 -- Homework  4 --  Due:  Oct 8, 2021}
\end{center}

\normalsize


\medskip

\noindent {\bf Practice Problems:}

\begin{pr}
    Dummit and Foote Section 4.5 problems 4, 19, 26, 30.
\end{pr}

\bigskip

Type solutions to the following problems in \LaTeX, and email the tex and PDF files to me at \url{dbernstein1@tulane.edu} by 10am on the due date.
Please title them as [lastname].tex and [lastname].pdf.
When preparing your solutions, you must follow the rules as laid out in the course syllabus.

\vspace{.5cm}

\noindent {\bf Graded Problems:}

\begin{pr}
    Solve the two following problems:
    \begin{enumerate}
        \item Use Sylow's Theorem to prove \emph{Cauchy's theorem}, which says that whenever $G$ is a group whose order is divisible by a prime $p$, then $G$ contains a subgroup of order $p$.
        \item Now, use Cauchy's theorem to prove that any \emph{abelian} group of order $pq$, with $p$ and $q$ prime, is cyclic.
    \end{enumerate}
\end{pr}

Given a field $\mathbb{F}$, recall that ${\rm GL}_n(\mathbb{F})$ is the group (under matrix multiplication) of $n\times n$ nonsingular matrices with entries in $\mathbb{F}$ and that $\mathbb{F}^{*}$ denotes the group, under multiplication, consisting of all nonzero elements of $\mathbb{F}$ (i.e.~$\mathbb{F}^{*} = {\rm GL}_1(\mathbb{F})$).
Recall that the map $\det: {\rm GL}_n(\mathbb{F}) \rightarrow \mathbb{F}^*$ sending a matrix to its determinant is a group homomorphism.

\begin{pr}
    Let $\phi: S_n \rightarrow {\rm GL}_n(\mathbb{Q})$ be the map sending a permutation $\sigma$ to the $n\times n$ matrix $\phi(\sigma)$ that has $1$ at the $(i,\sigma(i))$ entry for $1 = 1,\dots,n$, and $0$ otherwise.
    \begin{enumerate}
        \item Prove that $\phi$ is an injective group homomorphism. Matrices of the form $\phi(\sigma)$ are called \emph{permutation matrices}.
        \item Prove that every permutation matrix has determinant $\pm 1$.
        \item Given $\sigma \in S_n$, prove that $\det(\phi(\sigma)) = 1$ if and only if $\sigma \in A_n$.
    \end{enumerate}
\end{pr}





\end{document}