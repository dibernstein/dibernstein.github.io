\documentclass[letterpaper,11pt]{amsart}
\textwidth=16.00cm 
\textheight=22.00cm 
\topmargin=0.00cm
\oddsidemargin=0.00cm 
\evensidemargin=0.00cm 
\headheight=0cm 
\headsep=0.5cm

\textheight=610pt

\usepackage{latexsym,amsthm,amssymb,epsfig,url,tikz}%eufrak
%\usepackage[sumlimits]{amsmath}

\tikzstyle{vertex}=[circle, draw, inner sep=0pt, minimum size=6pt, fill]
\newcommand{\vertex}{\node[vertex]}

% \usepackage[centredisplay,PostScript=dvips]{diagrams}
%\usepackage[dvips]{color}
%\usepackage[mathscr]{eucal}mathrsfs

% Theorem Formatting Commands.
\theoremstyle{plain}
\newtheorem{thm}{Theorem}
%\newtheorem{lemma}{Lemma}[section]
\newtheorem{lemma}[thm]{Lemma}
\newtheorem{prop}[thm]{Proposition}
\newtheorem{cor}[thm]{Corollary}
\newtheorem{conj}[thm]{Conjecture}
\newtheorem*{thm*}{Theorem}
\newtheorem*{lemma*}{Lemma}
\newtheorem*{prop*}{Proposition}
\newtheorem*{cor*}{Corollary}
\newtheorem*{conj*}{Conjecture}

\theoremstyle{definition}
\newtheorem{defn}[thm]{Definition}
\newtheorem*{defn*}{Definition}
\newtheorem{ex}[thm]{Example}
\newtheorem{pr}{Problem}
\newtheorem{alg}[thm]{Algorithm}
\newtheorem{ques}[thm]{Question}

\theoremstyle{remark}
\newtheorem*{rmk}{Remark}
\newtheorem*{obs}{Observation}

\usepackage{color}
\newcommand{\dbfeedback}[1]{{\color{red} DB Feedback: #1}}

\newcommand{\conv}{{\rm Conv}}
\newcommand{\aff}{{\rm Aff}}
\newcommand{\rb}{{\rm rb}}
\newcommand{\ri}{{\rm relint}}
\newcommand{\rank}{{\rm rank}}



\begin{document}

\Large

\begin{center}
{\bf Math 7760 -- Homework  6 --  Due:  October 21, 2022}
\end{center}

\normalsize


\bigskip

\noindent {\bf Practice Problems:}

\bigskip

\begin{pr}
    If you are interested in learning about the connection between greedy algorithms and matroids, read Section 1.8 in Oxley.
\end{pr}

\begin{pr}
    Use matroid duality to prove the hard direction of the hyperplane axioms.
\end{pr}

\begin{pr}
    What is the dual of $U_{r,n}$?
\end{pr}

\bigskip

\noindent {\bf Problems to write up:}
\bigskip

\begin{pr}
    Let $M$ be a matroid with circuit $C$ and cocircuit $C^*$.
    Prove that $|C \cap C^*| \neq 1$.
\end{pr}

\begin{pr}
    Given a matroid $M$ on ground set $E$ and a basis $B$, prove that for each $e \in E\setminus B$,
    there exists a unique circuit in $B \cup \{e\}$.
    Use this to prove that for any connected graph $G$ with spanning tree $T$,
    for each $e \in T$ (this is not a typo), there exists a unique minimal cut of $G$ whose only edge in common with $T$ is $e$.
\end{pr}

\begin{pr}
    Let $A \in \mathbb{F}^{r \times n}$ have rank $r$
    and let $B \in \mathbb{F}^{(n-r) \times n}$ have rank $n-r$ and assume that $AB^T = 0$.
    Show that there exists a nonzero $\lambda \in \mathbb{F}$ such that for any $S \subseteq E$ of size $r$,
    if $A_S$ denotes the column-submatrix of $A$ on columns indexed by $S$ and $B_{\{1,\dots,n\}\setminus S}$
    denotes the column-submatrix of $B$ on columns indexed by $\{1,\dots,n\}\setminus S$,
    then $\det(A_S) = \lambda \det(B_{\{1,\dots,n\}\setminus S})$.
\end{pr}






\end{document}