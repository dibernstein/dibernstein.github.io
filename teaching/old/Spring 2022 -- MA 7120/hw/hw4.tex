 \documentclass[letterpaper,11pt]{amsart}
\textwidth=16.00cm 
\textheight=22.00cm 
\topmargin=0.00cm
\oddsidemargin=0.00cm 
\evensidemargin=0.00cm 
\headheight=0cm 
\headsep=0.5cm

\textheight=610pt

\usepackage{latexsym,amsthm,amssymb,epsfig,url}%eufrak
%\usepackage[sumlimits]{amsmath}

% \usepackage[centredisplay,PostScript=dvips]{diagrams}
%\usepackage[dvips]{color}
%\usepackage[mathscr]{eucal}mathrsfs

% Theorem Formatting Commands.
\theoremstyle{plain}
\newtheorem{thm}{Theorem}
%\newtheorem{lemma}{Lemma}[section]
\newtheorem{lemma}[thm]{Lemma}
\newtheorem{prop}[thm]{Proposition}
\newtheorem{cor}[thm]{Corollary}
\newtheorem{conj}[thm]{Conjecture}
\newtheorem*{thm*}{Theorem}
\newtheorem*{lemma*}{Lemma}
\newtheorem*{prop*}{Proposition}
\newtheorem*{cor*}{Corollary}
\newtheorem*{conj*}{Conjecture}

\theoremstyle{definition}
\newtheorem{defn}[thm]{Definition}
\newtheorem{ex}[thm]{Example}
\newtheorem{pr}{Problem}
\newtheorem{alg}[thm]{Algorithm}
\newtheorem{ques}[thm]{Question}

\theoremstyle{remark}
\newtheorem*{rmk}{Remark}
\newtheorem*{obs}{Observation}



\DeclareMathOperator{\gl}{GL}
\DeclareMathOperator{\aut}{Aut}


\usepackage{color}
\newcommand{\dbfeedback}[1]{{\color{red} DB Feedback: #1}}



\begin{document}

\Large

\begin{center}
{\bf Math 7120 -- Homework  4 --  Due:  February 23, 2022}
\end{center}

\normalsize

\medskip

\noindent {\bf Practice problems:}

\begin{pr}
    Prove that the functor $\hom_R(\_,D)$ is left-exact.
\end{pr}


\noindent {\bf Test prep:}

\begin{pr}
    Give an example of each of the following:
    \begin{enumerate}
        \item a projective module that is not free
        \item an injective $\mathbb{Z}$-module
        \item a non-injective $\mathbb{Z}$-module
    \end{enumerate}
\end{pr}




\bigskip

Type solutions to the following problems in \LaTeX, and email the tex and PDF files to me at \url{dbernstein1@tulane.edu} by 10am on the indicated date.
Please title them as [lastname].tex and [lastname].pdf.
When preparing your solutions, you must follow the rules as laid out in the course syllabus.

\vspace{.5cm}

\noindent {\bf Graded Problems:}

\begin{pr}
    Let $R$ be a ring with a $1$.
    Follow the prompts and hints laid out in Problems 15 and 16 in Section 10.5 to prove that
    that every $R$-module is contained in an injective $R$-module.
    You can cite any of the theorems in the book, aside from the implication $3 \implies 2$ of Proposition~34.
    Then, do exercise 17 in section 10.5.
\end{pr}








\end{document}