 \documentclass[letterpaper,11pt]{amsart}
\textwidth=16.00cm 
\textheight=22.00cm 
\topmargin=0.00cm
\oddsidemargin=0.00cm 
\evensidemargin=0.00cm 
\headheight=0cm 
\headsep=0.5cm

\textheight=610pt

\usepackage{latexsym,amsthm,amssymb,epsfig,url}%eufrak
%\usepackage[sumlimits]{amsmath}

% \usepackage[centredisplay,PostScript=dvips]{diagrams}
%\usepackage[dvips]{color}
%\usepackage[mathscr]{eucal}mathrsfs

% Theorem Formatting Commands.
\theoremstyle{plain}
\newtheorem{thm}{Theorem}
%\newtheorem{lemma}{Lemma}[section]
\newtheorem{lemma}[thm]{Lemma}
\newtheorem{prop}[thm]{Proposition}
\newtheorem{cor}[thm]{Corollary}
\newtheorem{conj}[thm]{Conjecture}
\newtheorem*{thm*}{Theorem}
\newtheorem*{lemma*}{Lemma}
\newtheorem*{prop*}{Proposition}
\newtheorem*{cor*}{Corollary}
\newtheorem*{conj*}{Conjecture}

\theoremstyle{definition}
\newtheorem{defn}[thm]{Definition}
\newtheorem{ex}[thm]{Example}
\newtheorem{pr}{Problem}
\newtheorem{alg}[thm]{Algorithm}
\newtheorem{ques}[thm]{Question}

\theoremstyle{remark}
\newtheorem*{rmk}{Remark}
\newtheorem*{obs}{Observation}



\DeclareMathOperator{\gl}{GL}
\DeclareMathOperator{\aut}{Aut}

\usepackage{color}
\newcommand{\dbfeedback}[1]{{\color{red} DB Feedback: #1}}



\begin{document}

\Large

\begin{center}
{\bf Math 7120 -- Homework  3 --  Due:  February 16, 2022}
\end{center}

\normalsize

\medskip

\noindent {\bf Practice problems:}

\begin{pr}
    Dummit and Foot 10.4 numbers 2,3,4
\end{pr}

\begin{pr}
    Prove that tensor product commutes with direct sum, i.e.~ if $M,M',N$ are $R$-modules, then
    \[
        (M \oplus M')\otimes_R N \cong (M\otimes_R N) \oplus (M' \otimes_R N).
    \]
    Conclude that if $S$ is an $\mathbb{R}$-algebra, then $S\otimes_R R^n \cong S^n$ as $R$-algebras.
\end{pr}



\noindent {\bf Test prep:}

\begin{pr}
    Simplify the following tensor products:
    \begin{enumerate}
        \item $\mathbb{Z}_4 \otimes_\mathbb{Z} \mathbb{Z}_2$
        \item $\mathbb{Z}_4 \otimes_{\mathbb{Z}} \mathbb{Z}$
        \item $\mathbb{Z}_4 \otimes_{\mathbb{Z}} \mathbb{Q}$
    \end{enumerate}
\end{pr}



\bigskip

Type solutions to the following problems in \LaTeX, and email the tex and PDF files to me at \url{dbernstein1@tulane.edu} by 10am on the indicated date.
Please title them as [lastname].tex and [lastname].pdf.
When preparing your solutions, you must follow the rules as laid out in the course syllabus.

\vspace{.5cm}

\noindent {\bf Graded Problems:}

\begin{pr}
    Let $R$ be a commutative ring with a $1$ and let $I,J$ be ideals of $R$.
    \begin{enumerate}
        \item Let $I = (2,x)$ be the ideal of the ring $R = \mathbb{Z}[x]$. Show that $2 \otimes 2 + x \otimes x$ is not a simple tensor in $I \otimes_R I$, i.e.~cannot be written in the form $a \otimes b$ for some $a,b \in I$.
        \item Prove that every element of $R/I \otimes_R R/J$ can be written as a simple tensor of the form $(1 \ {\rm mod} \ I) \otimes r \ {\rm mod} \ J$
        \item Prove that there is an $R$-module isomorphism $R/I \otimes_R R/J \cong R/(I+J)$ mapping $(r \ {\rm mod} \ I) \otimes (s \ {\rm mod} \ J)$ to $rs \ {\rm mod} \ (I+J)$.
        \item Suppose $R$ is an integral domain and $I$ is a principal ideal. Prove that $I \otimes_R I$ has no nonzero torsion elements.
    \end{enumerate}
\end{pr}

\begin{pr}
    Prove the part of the splitting lemma left as an exercise in class, i.e.~that if $0 \rightarrow A \overset{\psi}\rightarrow B \overset{\phi}\rightarrow C \rightarrow 0$ is a short exact sequence of $R$-modules and there exists $f: B \rightarrow A$ such that $f\circ \psi = {\rm id}_A$, then the sequence splits.
\end{pr}







\end{document}