\documentclass[letterpaper,11pt]{amsart}
\textwidth=16.00cm 
\textheight=22.00cm 
\topmargin=0.00cm
\oddsidemargin=0.00cm 
\evensidemargin=0.00cm 
\headheight=0cm 
\headsep=0.5cm

\textheight=610pt

\usepackage{latexsym,amsthm,amssymb,epsfig,url,tikz}%eufrak
%\usepackage[sumlimits]{amsmath}

\tikzstyle{vertex}=[circle, draw, inner sep=0pt, minimum size=6pt, fill]
\newcommand{\vertex}{\node[vertex]}

% \usepackage[centredisplay,PostScript=dvips]{diagrams}
%\usepackage[dvips]{color}
%\usepackage[mathscr]{eucal}mathrsfs

% Theorem Formatting Commands.
\theoremstyle{plain}
\newtheorem{thm}{Theorem}
%\newtheorem{lemma}{Lemma}[section]
\newtheorem{lemma}[thm]{Lemma}
\newtheorem{prop}[thm]{Proposition}
\newtheorem{cor}[thm]{Corollary}
\newtheorem{conj}[thm]{Conjecture}
\newtheorem*{thm*}{Theorem}
\newtheorem*{lemma*}{Lemma}
\newtheorem*{prop*}{Proposition}
\newtheorem*{cor*}{Corollary}
\newtheorem*{conj*}{Conjecture}

\theoremstyle{definition}
\newtheorem{defn}[thm]{Definition}
\newtheorem*{defn*}{Definition}
\newtheorem{ex}[thm]{Example}
\newtheorem{pr}{Problem}
\newtheorem{alg}[thm]{Algorithm}
\newtheorem{ques}[thm]{Question}

\theoremstyle{remark}
\newtheorem*{rmk}{Remark}
\newtheorem*{obs}{Observation}

\usepackage{color}
\newcommand{\dbfeedback}[1]{{\color{red} DB Feedback: #1}}

\newcommand{\conv}{{\rm Conv}}
\newcommand{\aff}{{\rm Aff}}
\newcommand{\rb}{{\rm rb}}
\newcommand{\ri}{{\rm relint}}
\newcommand{\rank}{{\rm rank}}



\begin{document}

\Large

\begin{center}
{\bf Math 7760 -- Homework  4 --  Due:  October 3, 2022}
\end{center}

\normalsize


\bigskip

\noindent {\bf Practice Problems:}

\bigskip

\begin{pr}
    Let $r,n$ be nonnegative integers with $r \le n$.
    Let $E$ be an $n$-element set and define $\mathcal{B}$ to be the set of all $r$-element subsets of $E$.
    \begin{enumerate}
        \item Convince yourself that $\mathcal{B}$ is the set of bases of a matroid $U_{r,n}$.
        Matroids of this form are called \emph{uniform matroids}.
        \item Determine what the independent sets, circuits, rank function, closure operator, and spanning sets of this matroid are.
    \end{enumerate}
\end{pr}

\begin{pr}
    Oxley, section 1.1 problems 1 and 4.
\end{pr}

\bigskip

\noindent {\bf Problems to write up:}
\bigskip

\begin{pr}
    Prove that $U_{2,n}$ is representable over a field with $q$ elements if and only if $q \ge n-1$.
    Does there exist a graphic matroid that is not representable over $\mathbb{F}_2$? Prove your answer.
\end{pr}

\begin{pr}
    Let $M$ be a binary matroid, i.e.~a matroid representable over the field with two elements.
    Prove that given any distinct circuits $C_1,C_2$, their symmetric difference $C_1 \Delta C_2$ contains a circuit.
\end{pr}

\bigskip





\end{document}