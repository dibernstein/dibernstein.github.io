 \documentclass[letterpaper,11pt]{amsart}
\textwidth=16.00cm 
\textheight=22.00cm 
\topmargin=0.00cm
\oddsidemargin=0.00cm 
\evensidemargin=0.00cm 
\headheight=0cm 
\headsep=0.5cm

\textheight=610pt

\usepackage{latexsym,amsthm,amssymb,epsfig,url}%eufrak
%\usepackage[sumlimits]{amsmath}

% \usepackage[centredisplay,PostScript=dvips]{diagrams}
%\usepackage[dvips]{color}
%\usepackage[mathscr]{eucal}mathrsfs

% Theorem Formatting Commands.
\theoremstyle{plain}
\newtheorem{thm}{Theorem}
%\newtheorem{lemma}{Lemma}[section]
\newtheorem{lemma}[thm]{Lemma}
\newtheorem{prop}[thm]{Proposition}
\newtheorem{cor}[thm]{Corollary}
\newtheorem{conj}[thm]{Conjecture}
\newtheorem*{thm*}{Theorem}
\newtheorem*{lemma*}{Lemma}
\newtheorem*{prop*}{Proposition}
\newtheorem*{cor*}{Corollary}
\newtheorem*{conj*}{Conjecture}

\theoremstyle{definition}
\newtheorem{defn}[thm]{Definition}
\newtheorem{ex}[thm]{Example}
\newtheorem{pr}{Problem}
\newtheorem{alg}[thm]{Algorithm}
\newtheorem{ques}[thm]{Question}

\theoremstyle{remark}
\newtheorem*{rmk}{Remark}
\newtheorem*{obs}{Observation}



\DeclareMathOperator{\gl}{GL}
\DeclareMathOperator{\aut}{Aut}


\usepackage{color}
\newcommand{\dbfeedback}[1]{{\color{red} DB Feedback: #1}}



\begin{document}

\Large

\begin{center}
{\bf Math 7120 -- Homework  7 --  Due:  March 21, 2022}
\end{center}

\normalsize

\medskip

Maybe: Dummit and Foote, 12.1 1-6 and 20

\noindent {\bf Practice problems:}

\begin{pr}
    Dummit and Foote, 12.2 problems 3,4,8
\end{pr}




\noindent {\bf Test prep:}

\begin{pr}
    Construct examples of each of the following, or briefly explain why it doesn't exist:
    \begin{enumerate}
        \item An $R$-module of rank one that is not free, where $R$ is an integral domain
        \item A nonzero free $F[t]$-module $M$ where $F$ is a field and $M$ is finite-dimensional as an $F$-vector space
        \item A non-Noetherian ring
        \item An infinitely-generated submodule of a finitely-generated $R$-module
    \end{enumerate}
\end{pr}


\begin{pr}
    Determine all possible rational canonical forms for a linear transformation $T: \mathbb{R}^6\rightarrow \mathbb{R}^6$ with characteristic polynomial $x^2(x^2+1)^2$.
\end{pr}

\begin{pr}
    Determine all possible Jordan canonical forms for a linear transformation $T: \mathbb{C}^5\rightarrow \mathbb{C}^5$ with characteristic polynomial $(x-2)^3(x-3)^2$.
\end{pr}






\bigskip

Type solutions to the following problems in \LaTeX, and email the tex and PDF files to me at \url{dbernstein1@tulane.edu} by 10am on the indicated date.
Please title them as [lastname].tex and [lastname].pdf.
When preparing your solutions, you must follow the rules as laid out in the course syllabus.

\vspace{.5cm}

\noindent {\bf Graded Problems:}


Recall that an $R$-module $M$ is \emph{torsion} if for each $x \in M$, there is $r \in R\setminus\{0\}$ such that $rx = 0$. 

\begin{pr}
    Let $R$ be an integral domain.
    \begin{enumerate}
        \item Prove that an $R$-module $M$ has rank $n$ if and only if $M$ contains a free submodule $N$ of rank $n$ such that $M/N$ is torsion.
        \item Prove that if $A$ and $B$ are $R$-modules with ranks $m$ and $n$, then $A \oplus B$ has rank $m+n$.
    \end{enumerate}
\end{pr}







\end{document}