 \documentclass[letterpaper,11pt]{amsart}
\textwidth=16.00cm 
\textheight=22.00cm 
\topmargin=0.00cm
\oddsidemargin=0.00cm 
\evensidemargin=0.00cm 
\headheight=0cm 
\headsep=0.5cm

\textheight=610pt

\usepackage{latexsym,amsthm,amssymb,epsfig,url}%eufrak
%\usepackage[sumlimits]{amsmath}

% \usepackage[centredisplay,PostScript=dvips]{diagrams}
%\usepackage[dvips]{color}
%\usepackage[mathscr]{eucal}mathrsfs

% Theorem Formatting Commands.
\theoremstyle{plain}
\newtheorem{thm}{Theorem}
%\newtheorem{lemma}{Lemma}[section]
\newtheorem{lemma}[thm]{Lemma}
\newtheorem{prop}[thm]{Proposition}
\newtheorem{cor}[thm]{Corollary}
\newtheorem{conj}[thm]{Conjecture}
\newtheorem*{thm*}{Theorem}
\newtheorem*{lemma*}{Lemma}
\newtheorem*{prop*}{Proposition}
\newtheorem*{cor*}{Corollary}
\newtheorem*{conj*}{Conjecture}

\theoremstyle{definition}
\newtheorem{defn}[thm]{Definition}
\newtheorem{ex}[thm]{Example}
\newtheorem{pr}[thm]{Problem}
\newtheorem{alg}[thm]{Algorithm}
\newtheorem{ques}[thm]{Question}

\theoremstyle{remark}
\newtheorem*{rmk}{Remark}
\newtheorem*{obs}{Observation}



\usepackage{color}
\newcommand{\dbfeedback}[1]{{\color{red} DB Feedback: #1}}



\begin{document}

\Large

\begin{center}
{\bf Math 7120 -- Homework  6 --  Due:  March 11, 2022}
\end{center}

\normalsize

\medskip


The punchline of this worksheet is that the exterior algebra of a vector space is, in a certain sense, an algebra of linear subspaces. {\bf This homework will not be graded.}

% \begin{defn}
%     Let $\mathbb{K}$ be a field and let $R,S$ be $\mathbb{K}$-algebras.
%     A \emph{$\mathbb{K}$-algebra homomorphism} is a ring homomorphism $\phi: R\rightarrow S$
%     such that $\phi$ is linear as a map of $\mathbb{K}$-vector spaces.
% \end{defn}

\begin{pr}
    Let $\mathbb{K}$ be a field and let
    $V,W$ be finite dimensional $\mathbb{K}$-vector spaces.
    Let $T: V \rightarrow W$ be a linear transformation.
    Suppose that $\{e_i\}$ and $\{f_j\}$ are bases of $V$ and $W$
    and let $A$ denote the matrix of $T$ map with respect to these bases.
    Let $A_{i_1,\dots,i_M;j_1,\dots,j_M}$ denote the determinant of the $M\times M$ matrix obtained from $A$ by restricting to the rows $i_1,\dots,i_M$ and columns $j_1,\dots,j_M$.
    Define a linear map
    \[
        \bigwedge T : \bigwedge V \rightarrow \bigwedge W
    \]
    by extending the following function linearly to all of $\bigwedge V$
    \[
        e_{j_1}\wedge \dots \wedge e_{j_M} \mapsto \sum_{i_1 < \dots < i_M} A_{i_1,\dots,i_M;j_1,\dots,j_M} f_{i_1}\wedge \dots \wedge f_{i_M}.
    \]
    Verify that $\bigwedge T$ is a $\mathbb{K}$-linear ring homomorphism.
    Show that if $v_1,\dots,v_M \in V$ are linearly independent and $w_1,\dots,w_M \in {\rm span}\{v_1,\dots,v_M\}$ with $w_i = \sum_{j=1}^M a_{ij}v_j$, then, letting $A$ be the matrix whose $ij$ entry is $a_{ij}$, the following holds
        \[
            w_1\wedge\dots\wedge w_M = \det(A) v_1 \wedge \dots \wedge v_M.
        \]
        % \item If $T: V\rightarrow V$ is linear and $V$ has dimension $N$, then $\bigwedge T$ maps $\bigwedge^N V$ to itself
        % and is just multiplication by $\det(T)$.
        % \item If $T: V \rightarrow W$ is linear, then ${\rm rank}(T) < M$ if and only if $\bigwedge T(\bigwedge^M V) = 0$. In particular, a matrix has rank less than $M$ if and only if all $M\times M$ submatrices have zero determinant.
\end{pr}

\begin{defn}
    Let $\omega \in \bigwedge^m V$. Then
    \begin{enumerate}
        \item $\omega$ is \emph{completely decomposable} if there exist $v_1,\dots,v_m$ such that $\omega = v_1 \wedge \dots \wedge v_m$, and
        \item $\omega$ is \emph{partially decomposable} if $\omega = v \wedge \eta$ for some $v \in V$ and $\eta \in \bigwedge^{m-1} V$.
    \end{enumerate}
    Note that if $w$ is completely decomposable, it is also partially decomposable.
\end{defn}

\begin{pr}
    Let $\omega \in \bigwedge^m V$ and define $\phi_\omega: V\rightarrow \bigwedge^{m+1}V$ by $v\mapsto v\wedge \omega$.
    Prove the following.
    \begin{enumerate}
        \item If $\omega$ is partially decomposable, then $\omega \wedge \omega = 0$.
        \item $\omega$ is partially decomposable if and only if $\phi_\omega$ has nontrivial kernel.
        \item If $\{v_1,\dots,v_M\}$ is a basis for the kernel of $\phi_\omega$, then there exists $\eta \in \bigwedge^{m-M} V$ such that
        \[
            \omega = v_1 \wedge \dots \wedge v_M \wedge \eta.
        \]
        \item $\omega$ is completely decomposable if and only if the kernel of $\phi_\omega$ has dimension $m$.
    \end{enumerate}
\end{pr}

\begin{defn}
    Given elements $u,v$ of an $\mathbb{K}$-vector space $W$, we say that $u,v$ are \emph{protectively equivalent} if there exists a nonzero scalar $\lambda \in \mathbb{K}$ such that $u = \lambda v$.
\end{defn}

\begin{pr}
    Let $W$ be a vector space of dimension $N$ and let $M \le N$.
    For each vector subspace $V$ of $W$ of dimension $M$, define $j(V)$ to be the projective equivalence class of
    $v_1 \wedge \dots \wedge v_M$ where $v_1,\dots,v_M$ is a basis of $V$.
    Prove the following:
    \begin{enumerate}
        \item $j$ is a well-defined map (i.e.~does not depend on choice of basis),
        \item $j$ is a bijection between linear subspaces of $W$ of dimension $M$, and projective equivalence classes of completely decomposable elements of $\bigwedge^M W$, and
        \item if $V_1,V_2$ are subspaces of $W$ such that $V_1\cap V_2 = \{0\}$, then $j(V_1 + V_2) = j(V_1)\wedge j(V_2)$.
    \end{enumerate}
\end{pr}








\end{document}