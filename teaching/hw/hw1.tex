 \documentclass[letterpaper,11pt]{amsart}
\textwidth=16.00cm 
\textheight=22.00cm 
\topmargin=0.00cm
\oddsidemargin=0.00cm 
\evensidemargin=0.00cm 
\headheight=0cm 
\headsep=0.5cm

\textheight=610pt

\usepackage{latexsym,amsthm,amssymb,epsfig,url}%eufrak
%\usepackage[sumlimits]{amsmath}

% \usepackage[centredisplay,PostScript=dvips]{diagrams}
%\usepackage[dvips]{color}
%\usepackage[mathscr]{eucal}mathrsfs

% Theorem Formatting Commands.
\theoremstyle{plain}
\newtheorem{thm}{Theorem}
%\newtheorem{lemma}{Lemma}[section]
\newtheorem{lemma}[thm]{Lemma}
\newtheorem{prop}[thm]{Proposition}
\newtheorem{cor}[thm]{Corollary}
\newtheorem{conj}[thm]{Conjecture}
\newtheorem*{thm*}{Theorem}
\newtheorem*{lemma*}{Lemma}
\newtheorem*{prop*}{Proposition}
\newtheorem*{cor*}{Corollary}
\newtheorem*{conj*}{Conjecture}

\theoremstyle{definition}
\newtheorem{defn}[thm]{Definition}
\newtheorem{ex}[thm]{Example}
\newtheorem{pr}{Problem}
\newtheorem{alg}[thm]{Algorithm}
\newtheorem{ques}[thm]{Question}

\theoremstyle{remark}
\newtheorem*{rmk}{Remark}
\newtheorem*{obs}{Observation}



\DeclareMathOperator{\gl}{GL}
\DeclareMathOperator{\aut}{Aut}

\usepackage{color}
\newcommand{\dbfeedback}[1]{{\color{red} DB Feedback: #1}}



\begin{document}

\Large

\begin{center}
{\bf Math 7120 -- Homework  1 --  Due:  February 2, 2022}
\end{center}

\normalsize



\medskip

\noindent {\bf Practice problems:}

\begin{pr}
    Dummit and Foote, 10.1 problems 1 and 3.
\end{pr}

\noindent {\bf Test prep:}

\begin{pr}
    Let $n \ge 2$ be an integer and let $R$ be the ring of $n\times n$ matrices with entries in $\mathbb{C}$.
    Let $M_l$ and $M_r$ be the module structures on $R$ defined by left and right multiplication (i.e.~given $r,m \in R$, then multiplying $r$ by $m$ is $rm$ when considering $m$ as an element of $M_l$, and $mr$ when considering $m$ as an element of $M_r$).
    Let $Z$ be the subset of $R$ consisting of all matrices whose last $n-1$ columns are zero.
    Determine whether $N$ is a submodule of $M_l$ and $M_r$.
\end{pr}

\begin{pr}
    Determine which of the statements below are true. For those that are false, provide a counterexample:
    \begin{enumerate}
        \item if $R$ is a subring of $S$ and $M$ is an $S$-module, then $M$ is an $R$-module
        \item if $R$ is a subring of $S$ and $M$ is an $R$-module, then $M$ is an $S$-module.
    \end{enumerate}
\end{pr}

\bigskip

Type solutions to the following problems in \LaTeX, and email the tex and PDF files to me at \url{dbernstein1@tulane.edu} by 10am on the indicated date.
Please title them as [lastname].tex and [lastname].pdf.
When preparing your solutions, you must follow the rules as laid out in the course syllabus.

\vspace{.5cm}

\noindent {\bf Graded Problems:}

\begin{pr}
    The \emph{annihilator} of an $R$-module $M$ is defined to be
    \[
        \{r \in R : rm = 0 \ {\rm for \ all} \ m \in M\}.
    \]
    \begin{enumerate}
        \item Prove that the annihilator of an $R$-module is a two-sided ideal of $R$.
        \item Let $M$ be a finitely generated abelian group, viewed as a $\mathbb{Z}$ module. What is the annihilator of $M$?
    \end{enumerate}
\end{pr}

\begin{pr}
    Let $F = \mathbb{R}$ and $V = \mathbb{R}^2$.
    Recall that each linear transformation $T: V \rightarrow V$ gives rise to an $F[t]$-module.
    For each of the following values of $T$, determine all $F[t]$-submodules of $V$:
    \begin{enumerate}
        \item $T$ is $90^\circ$ clockwise rotation about the origin
        \item $T$ is orthogonal projection onto the $y$-axis
        \item $T$ is $180^\circ$ clockwise rotation about the origin.
    \end{enumerate}
\end{pr}









\end{document}