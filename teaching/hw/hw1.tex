 \documentclass[letterpaper,11pt]{amsart}
\textwidth=16.00cm 
\textheight=22.00cm 
\topmargin=0.00cm
\oddsidemargin=0.00cm 
\evensidemargin=0.00cm 
\headheight=0cm 
\headsep=0.5cm

\textheight=610pt

\usepackage{latexsym,amsthm,amssymb,epsfig,url}%eufrak
%\usepackage[sumlimits]{amsmath}

% \usepackage[centredisplay,PostScript=dvips]{diagrams}
%\usepackage[dvips]{color}
%\usepackage[mathscr]{eucal}mathrsfs

% Theorem Formatting Commands.
\theoremstyle{plain}
\newtheorem{thm}{Theorem}
%\newtheorem{lemma}{Lemma}[section]
\newtheorem{lemma}[thm]{Lemma}
\newtheorem{prop}[thm]{Proposition}
\newtheorem{cor}[thm]{Corollary}
\newtheorem{conj}[thm]{Conjecture}
\newtheorem*{thm*}{Theorem}
\newtheorem*{lemma*}{Lemma}
\newtheorem*{prop*}{Proposition}
\newtheorem*{cor*}{Corollary}
\newtheorem*{conj*}{Conjecture}

\theoremstyle{definition}
\newtheorem{defn}[thm]{Definition}
\newtheorem*{defn*}{Definition}
\newtheorem{ex}[thm]{Example}
\newtheorem{pr}{Problem}
\newtheorem{alg}[thm]{Algorithm}
\newtheorem{ques}[thm]{Question}

\theoremstyle{remark}
\newtheorem*{rmk}{Remark}
\newtheorem*{obs}{Observation}

\usepackage{color}
\newcommand{\dbfeedback}[1]{{\color{red} DB Feedback: #1}}



\begin{document}

\Large

\begin{center}
{\bf Math 7760 -- Homework  1 --  Due:  August 31, 2022}
\end{center}

\normalsize


\medskip

\noindent {\bf Practice Problems:}

\begin{pr}
    Show that every compact convex set has an extreme point. Give an example of a non-compact convex set with an extreme point.
\end{pr}

\begin{pr}
    Let $P,Q \subset \mathbb{R}^2$ be two-dimensional polytopes (i.e.~polygons).
    For each of the following statements, either prove that they are true, or provide a counterexample.
    \begin{enumerate}
        \item If $P$ and $Q$ have the same number of edges, then they are affinely isomorphic.
        \item If $P$ and $Q$ have the same number of edges, then they are combinatorially isomorphic.
        \item If $P$ and $Q$ are both triangles, then they are affinely isomorphic.
        \item If $P$ is a square and $Q$ is a parallelogram, then $P$ and $Q$ are affinely isomorphic.
        Begin by convincing yourself that it makes no difference if you assume that the vertices of $P$ are $\{0,1\}^2$ and that
        $(0,0)$ is a vertex of $Q$.
    \end{enumerate}
\end{pr}



\bigskip

Type solutions to the following problems in \LaTeX, and email the tex and PDF files to me at \url{dbernstein1@tulane.edu} on the due date.
Please title them as [lastname].tex and [lastname].pdf.
When preparing your solutions, you must follow the rules as laid out in the course syllabus.

\vspace{.5cm}

\noindent {\bf Graded Problems:}







\end{document}