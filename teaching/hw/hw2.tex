 \documentclass[letterpaper,11pt]{amsart}
\textwidth=16.00cm 
\textheight=22.00cm 
\topmargin=0.00cm
\oddsidemargin=0.00cm 
\evensidemargin=0.00cm 
\headheight=0cm 
\headsep=0.5cm

\textheight=610pt

\usepackage{latexsym,amsthm,amssymb,epsfig,url}%eufrak
%\usepackage[sumlimits]{amsmath}

% \usepackage[centredisplay,PostScript=dvips]{diagrams}
%\usepackage[dvips]{color}
%\usepackage[mathscr]{eucal}mathrsfs

% Theorem Formatting Commands.
\theoremstyle{plain}
\newtheorem{thm}{Theorem}
%\newtheorem{lemma}{Lemma}[section]
\newtheorem{lemma}[thm]{Lemma}
\newtheorem{prop}[thm]{Proposition}
\newtheorem{cor}[thm]{Corollary}
\newtheorem{conj}[thm]{Conjecture}
\newtheorem*{thm*}{Theorem}
\newtheorem*{lemma*}{Lemma}
\newtheorem*{prop*}{Proposition}
\newtheorem*{cor*}{Corollary}
\newtheorem*{conj*}{Conjecture}

\theoremstyle{definition}
\newtheorem{defn}[thm]{Definition}
\newtheorem*{defn*}{Definition}
\newtheorem{ex}[thm]{Example}
\newtheorem{pr}{Problem}
\newtheorem{alg}[thm]{Algorithm}
\newtheorem{ques}[thm]{Question}

\theoremstyle{remark}
\newtheorem*{rmk}{Remark}
\newtheorem*{obs}{Observation}

\usepackage{color}
\newcommand{\dbfeedback}[1]{{\color{red} DB Feedback: #1}}

\newcommand{\conv}{{\rm Conv}}
\newcommand{\aff}{{\rm Aff}}
\newcommand{\rb}{{\rm rb}}
\newcommand{\ri}{{\rm relint}}
\newcommand{\rank}{{\rm rank}}



\begin{document}

\Large

\begin{center}
{\bf Math 7760 -- Homework  2 --  Due:  September 7, 2022}
\end{center}

\normalsize


\bigskip

\noindent {\bf Practice Problems:}

\bigskip

\begin{pr}
    Prove that $\dim(\conv\{v_1,\dots,v_n\}) = \rank(\hat{V}) - 1$ where
    \[
        \hat{V} = \begin{pmatrix}
            1 & 1 & \dots & 1 \\
            v_1 & v_2 & \dots & v_n
        \end{pmatrix}
    \]
\end{pr}

\begin{pr}\label{convexNearest}
    Recall from analysis that if $S \subseteq \mathbb{R}^n$ is compact, then each continuous function $f: S \rightarrow R$
    has a maximum and a minimum on $S$.
    Now let $C\subseteq \mathbb{R}^d$ be closed and convex and let $x \in \mathbb{R}^d \setminus C$.
    Prove that there exists a unique point $y \in C$ minimizing the Euclidean distance to $x$.
    [Hint: begin by reducing to the case that $C$ is compact.]
\end{pr}

\bigskip

\noindent {\bf Problems to write up:}

\bigskip

\begin{pr}
    Prove the hyperplane separation theorem (see below for the theorem statement and a proof outline).
\end{pr}

\begin{thm*}[Hyperplane separation theorem]
    Given a convex $C\subset \mathbb{R}^n$ and a point $y \in \mathbb{R}^d\setminus C$,
    there exists $a \in (\mathbb{R}^n)^*$ and $b \in \mathbb{R}$ such that $ax \le b$ for all $x \in C$ and $ay \ge b$.
\end{thm*}
\noindent {\bf Outline of proof:}\\
    Split into the cases based on whether or not $y \in \rb(C)$.\\
    {\bf Case 1:} $y \notin \rb(C)$:
    \begin{enumerate}
        \item Prove that there exists a unique $z \in C$ that is nearest
        to $y$ in the Euclidean distance metric (c.f.~Problem~\ref{convexNearest}).
        \item Reduce to the case where $z = -y$.
        \item Show that $y^Tx \le 0$ for all $x \in C$. [Hint: show that if $y^Tx > 0$,
        then $tx + (1-t)z$ and $y$ are closer to each other than $z$ and $y$ are to each other for small positive $t$.]
        \item Conclude that the hyperplane separation theorem is true when $y \notin \rb(C)$.
    \end{enumerate}
    {\bf Case 2:} $y \in \rb(C)$:
    \begin{enumerate}
        \item Reduce to the case that $y = 0$ and note that it is enough to construct a \emph{linear} hyperplane
        that does not intersect $\ri(C)$.
        \item Inductively construct a sequence of linear spaces $L_0,\dots,L_{d-1} \subseteq \mathbb{R}^d$ with $\dim L_i = i$,
        none intersecting $\ri(C)$. [Hint: for $k \le d-2$, note that $L_k^\perp$ has a two-dimensional subspace $P$
        and that $P \cap (C + L_k)$ is convex.
        Argue that $P$ contains a line $L$ through the origin that does not intersect $P \cap (\ri(C) + L_k)$, and that this implies $L_k + L$ does not intersect $\ri(C)$.]
        \item Conclude that the hyperplane separation theorem is true.
    \end{enumerate}






\end{document}