 \documentclass[letterpaper,11pt]{amsart}
\textwidth=16.00cm 
\textheight=22.00cm 
\topmargin=0.00cm
\oddsidemargin=0.00cm 
\evensidemargin=0.00cm 
\headheight=0cm 
\headsep=0.5cm

\textheight=610pt

\usepackage{latexsym,amsthm,amssymb,epsfig,url}%eufrak
%\usepackage[sumlimits]{amsmath}

% \usepackage[centredisplay,PostScript=dvips]{diagrams}
%\usepackage[dvips]{color}
%\usepackage[mathscr]{eucal}mathrsfs

% Theorem Formatting Commands.
\theoremstyle{plain}
\newtheorem{thm}{Theorem}
%\newtheorem{lemma}{Lemma}[section]
\newtheorem{lemma}[thm]{Lemma}
\newtheorem{prop}[thm]{Proposition}
\newtheorem{cor}[thm]{Corollary}
\newtheorem{conj}[thm]{Conjecture}
\newtheorem*{thm*}{Theorem}
\newtheorem*{lemma*}{Lemma}
\newtheorem*{prop*}{Proposition}
\newtheorem*{cor*}{Corollary}
\newtheorem*{conj*}{Conjecture}

\theoremstyle{definition}
\newtheorem{defn}[thm]{Definition}
\newtheorem{ex}[thm]{Example}
\newtheorem{pr}{Problem}
\newtheorem{alg}[thm]{Algorithm}
\newtheorem{ques}[thm]{Question}

\theoremstyle{remark}
\newtheorem*{rmk}{Remark}
\newtheorem*{obs}{Observation}



\DeclareMathOperator{\gl}{GL}
\DeclareMathOperator{\aut}{Aut}

\usepackage{color}
\newcommand{\dbfeedback}[1]{{\color{red} DB Feedback: #1}}



\begin{document}

\Large

\begin{center}
{\bf Math 7120 -- Homework  2 --  Due:  February 9, 2022}
\end{center}

\normalsize

\medskip

\noindent {\bf Practice problems:}

\begin{pr}
    Dummit and Foote 10.2 problems 4 and 6.
\end{pr}

\begin{pr}
    Read the proof of the universal theorem for free modules (Theorem 6 in section 10.3).
\end{pr}

\noindent {\bf Test prep:}

\begin{pr}
    Dummit and Foote 10.2 problems 3 and 5.
\end{pr}


\bigskip

Type solutions to the following problems in \LaTeX, and email the tex and PDF files to me at \url{dbernstein1@tulane.edu} by 10am on the indicated date.
Please title them as [lastname].tex and [lastname].pdf.
When preparing your solutions, you must follow the rules as laid out in the course syllabus.

\vspace{.5cm}

\noindent {\bf Graded Problems:}


\begin{pr}
    An $R$-module $M$ is a \emph{torsion} module if for all $m\in M$ there exists a nonzero $r \in R$ such that $rm = 0$.
    \begin{enumerate}
        \item Prove that every finite abelian group is a torsion $\mathbb{Z}$-module.
        \item If $G$ is an infinite abelian group, is it necessarily true that $G$ is \emph{not} a torsion $\mathbb{Z}$-module?
        \item Let $R$ be an integral domain. Prove that if $M$ is a finitely generated torsion $R$-module, then $M$ has a nonzero annihilator (see previous HW for definition of annihilator).
        \item Give an example of a ring $R$ and a torsion $R$-module $M$ such that the annihilator of $M$ is the zero ideal of $R$.
    \end{enumerate}
\end{pr}

\begin{pr}
    An $R$-module is \emph{irreducible} if $M \neq 0$ and if $0$ and $M$ are the only submodules.
    \begin{enumerate}
        \item Show that $M$ is irreducible if and only if $M \neq 0$ and $M$ is a cyclic module with any nonzero element as a generator.
        \item Prove the following fundamental result of representation theory, often known as \emph{Schur's lemma}, which says that if $M_1$ and $M_2$ are irreducible $R$-modules, then any nonzero $R$-module homomorphism $M_1\rightarrow M_2$ is an isomorphism.
        \item Show that if $M$ is irreducible, then ${\rm Hom}_R(M,M)$ is a division ring.
        \item Assume $R$ is commutative. Show that an $R$-module $M$ is irreducible if and only if $M$ is isomorphic (as an $R$-module) to $R/I$ where $I$ is a maximal ideal of $R$.
        \item Determine all irreducible $\mathbb{Z}$-modules.
    \end{enumerate}
\end{pr}









\end{document}